\chapter{Conclusion}


In this thesis, we have progressed the MLinvitroTox framework to predict the toxicity of unidentified chemical compounds found in environmental samples, with the objective of guiding the prioritization process for identifying the most hazardous substances as part of an early-stage monitoring effort. In this context, our objective was to enhance the ability to detect toxic compounds while reducing the occurrence of false positives. Our ML models have demonstrated their potential to predict toxicity across 345 target assay endpoints sourced from the ToxCast dataset. These predictions rely on molecular fingerprints derived from both the chemical structure and SIRIUS-predicted fingerprints generated from MassBank spectra.
The performance outcomes were thoroughly investigated across various combinations of models and metrics, providing a comprehensive understanding of the strengths and weaknesses of binarized classification used in hazard assessment. It became evident that the selection of the classification threshold and class metric plays a pivotal role in striking a balance between identifying toxic compounds and minimizing false alarms, especially when dealing with highly imbalanced toxicity datasets.
To summarize, although our pipeline shows promise, there remains ample room for improvement, and further development is essential to assess its applicability in real-world scenarios.

\section{Future Work}
Given the relatively modest performance of our models, we aim to delve into the underlying reasons and explore avenues for improvement. One promising area for further exploration involves refining our approach to cytotoxicity assessment, which includes carefully curating hitcalls to create cleaner target variables that exclusively capture specific toxicity.
The probabilistic curation strategy employed in this thesis yielded inferior model performance when compared to toxicity target variables derived solely from curve fitting and hit-calling. A major drawback of our approach was that it solely has taken into accound critical concentrations at which cytotoxicity and specific activity were observed without considering the magnitude of these toxicities. Consequently, we lack estimates of the proportions of cytotoxic bursts and specific toxicity, which may have led to the overly aggressive filtering, i.e., downgrading toxic compounds to non-toxic. A more fine-grained approach would be beneficial.

Additionally, it would be worthwile to explore the coverage of the chemical space that could offer valuable insights into the distribution of training and validation sets, potentially contributing to our understanding of the model's performance.

Moreover, for the application of this framework it would be vital to exclude assay endpoints with poor validation performance when it comes to predicting the final toxicity fingerprints of the pipelined environmental sample inputs. Currently, the toxicity fingerprint for a compound relies on the aggregation of toxicity predictions from all assay endpoints, which can introduce potential inaccuracies if endpoints with unsatisfactory predictive ability are included. 

Furthermore, investigating the impact of utilizing particular aggregation strategies to improve the informativeness of the final toxicity fingerprints could be valuable. However, this endeavor would require a profound grasp of the biochemical processes that underlie the assay endpoints, along with their corresponding interpretations in the annotations. It should be undertaken in alignment with the overarching project objectives and its intended application.
