\chapter{Conclusion}

We have evidence of a multitude of chemicals being present in the environment and in our bodies and that mixture exposure indeed matters. This knowledge needs to be deepened, and the quantitative contribution of chemicals to compromised health should be better described and translated into regulatory action. As indicated in a scientific opinion paper of the German Federal Environmental Agency (Conrad et al. 2021), the CSS goals may be considered as a moving target. For increasing scientific evidence and improved method for detection and assessment of chemicals, development of new technologies require innovative regulatory, technological and societal reactions. We should be flexible and prepared to take up the scientific challenges and collaborate productively with regulatory institutions to address the identified challenges and modernise chemical risk assessment. This is also in line with the concern of many scientists that chemical pollution and the wide range of adverse effects on human and ecosystem health demand additional efforts on a global scale (Brack et al. 2022; Wang et al. 2021). We see the CSS as a European strategy that, in concert with other initiatives, may open new opportunities to minimise hazardous chemical pollution and thus risks to human health and ecosystems.


\section{Further}
mistures are not combination of mixtures toxic effects are not tested and produce an exponential times more assessment work