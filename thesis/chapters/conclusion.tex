\chapter{Conclusion}

We have evidence of a multitude of chemicals being present in the environment and in our bodies and that mixture exposure indeed matters. This knowledge needs to be deepened, and the quantitative contribution of chemicals to compromised health should be better described and translated into regulatory action. As indicated in a scientific opinion paper of the German Federal Environmental Agency (Conrad et al. 2021), the CSS goals may be considered as a moving target. For increasing scientific evidence and improved method for detection and assessment of chemicals, development of new technologies require innovative regulatory, technological and societal reactions. We should be flexible and prepared to take up the scientific challenges and collaborate productively with regulatory institutions to address the identified challenges and modernise chemical risk assessment. This is also in line with the concern of many scientists that chemical pollution and the wide range of adverse effects on human and ecosystem health demand additional efforts on a global scale (Brack et al. 2022; Wang et al. 2021). We see the CSS as a European strategy that, in concert with other initiatives, may open new opportunities to minimise hazardous chemical pollution and thus risks to human health and ecosystems.


\section{Further}
mistures are not combination of mixtures toxic effects are not tested and produce an exponential times more assessment work

Nevertheless, when it comes to the selection of chemicals and coverage of chemical space,  representative of the wide coverage of natural and anthropogenic chemicals detectable with nontarget LC-HRMS. Due to the limited availability of such toxicity data and a likely bias in the existing database toward toxic substances, some of the relations between toxicity and mass spectrometry variables may also be biased.


Additionally, MS2Tox uses the structural fingerprints predicted from MS2 spectra with SIRIUS+CSI:FingerID software, and the accuracy of MS2Tox is therefore related to the accuracy of these fingerprint predictions. In the case of poor MS2 data (low number of fragment peaks, noisy spectrum), incorrectly calculated fingerprints can strongly affect the predicted toxicity.

In the future, we envision that it will become possible to advance MS2Tox for evaluating the toxic effects of complex mixtures, such as those present in drinking water or wastewater, as well as the contributing effect of individual compounds in these samples. This rapid knowledge will be essential for identifying toxic substances and mitigating environmental harm in real-world scenarios, such as for effluents. No additional measurements by nontarget LC-HRMS should be required, and laboratories analyzing water samples with LC-HRMS can use MS2Tox without additional equipment for toxicity testing or pure test compounds. Together with concentration predictions, (53) MS2Tox might be used to evaluate the hazard of different waters as well as prioritize chemicals for identification and removal.