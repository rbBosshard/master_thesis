\chapter{Literature Review}\label{sec:literature}
\section{Background}
\section{Context}

In their systematic investigation using Tox21 data, Wu et al. (2021) explored the impact of various modeling approaches and chemical features on predictive toxicology, with a focus on model performance and explainability trade-offs. The study found that the assay endpoint from the Tox21 data being predicted was the most significant factor influencing model performance. Endpoints with higher predictability, characterized by lower data imbalance and larger datasets, performed well regardless of the modeling approach or molecular representation. For less predictable endpoints, simpler models like Linear Regression performed similarly to complex ones, prioritizing both predictivity and interpretability. Moreover this study suggests consensus modeling and multi-task learning to enhance predictability and model performance across endpoints. In this thesis, we set the goal to not overlook simpler models due to their higher interpretability and comparable performance. As suggested we do not further investigate on the differnt molecular representations and use a fixed compilation of molecolar fingerprints (Sirius) as initial input features. We incorporated in our studies a form of consensus modeling to consolidate predictability and multi-task learning to improve model performance across different endpoints.
