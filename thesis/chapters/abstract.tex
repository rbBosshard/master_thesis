\begin{abstract}
  This thesis enhances the capabilities of the MLinvitroTox framework, which is designed to forecast the toxicity of unidentified chemical compounds utilizing High-Resolution Mass Spectrometry (HRMS/MS) data. The framework aims to maximize the detection rate of most hazardous compounds within environmental samples, all while minimizing the occurrence of false alarms, thus channeling resources and efforts into the labor-intensive task of identifying and quantifying compounds that possess the highest potential for causing harm. This approach stands in contrast to standard nontarget screening HRMS workflows, which typically prioritize compounds that are detected most frequently and with highest intensities in the mass-to-charge ratio of ions in a sample. We employed hazard-driven machine learning models based on molecular fingerprints derived from chemical structure and utilized \emph{in vitro} toxicity data from ToxCast/Tox21. We have developed \texttt{pytcpl}, a \texttt{Python} processing pipeline that is applicable to the latest toxicity data. We have leveraged datasets spanning various assay endpoints that encompass a wide range of toxicity aspects. The XGBoost classifier achieves a median F1-score of 0.65 across all 345 target assay endpoints when predicting binary toxicity based on molecular fingerprints from known chemical structures. The models also demonstrate their predictive effectiveness when validated on SIRIUS predicted molecular fingerprints from MassBank spectra. Furthermore, a web app was created to facilitate interaction with the data and the MLinvitroTox framework.
\end{abstract}
