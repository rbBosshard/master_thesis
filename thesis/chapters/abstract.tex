\begin{abstract}
  This thesis enhances the MLinvitroTox framework, which predicts the toxicity of unknown compounds from High-Resolution Mass-Spectrometry (HRMS/MS) fragmentation spectra data. This framework forecasts the most hazardous compounds in environmental samples, circumventing the need for resource-intensive chemical identification. The predictivity is evaluated on SIRIUS molecular fingerprints from MassBank spectra data. However, the machine learning models are trained on molecular fingerprints from structure and uses in vitro toxicity data from ToxCast/Tox21. We have developed pytcpl, a Python-based processing pipeline that extends its applicability to the latest toxicity data. We have employed datasets for various assay endpoints, encompassing diverse aspects of toxicity. The individual machine learning models achieve an average balanced accuracy of todo:X when predicting binary toxicity, and they also demonstrate effectiveness when validated using MassBank spectra data. Furthermore, we've created a user-friendly web app to facilitate interaction with this framework.
\end{abstract}
