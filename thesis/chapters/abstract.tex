\begin{abstract}
  This thesis enhances the MLinvitroTox framework, which predicts the toxicity of unknown chemical compounds from High-Resolution Mass Spectrometry (HRMS/MS) data. The framework predicts the most hazardous compounds within environmental samples, channeling resources and efforts into the labor-intensive task of identifying and quantifying compounds with the highest potential for causing harm. This approach contrasts with the common practice in standard nontarget screening HRMS workflows, which typically prioritizes compounds based on their frequent detection and high intensities in the spectral data. The machine learning models are trained on molecular fingerprints derived from chemical structure and utilized \emph{in vitro} toxicity data from ToxCast/Tox21. We have developed pytcpl, a Python-based processing pipeline that is applicable to the latest toxicity data. We have leveraged datasets spanning various assay endpoints that encompass a wide range of toxicity aspects. The individual machine learning models achieve an average balanced accuracy of todo:X when predicting binary toxicity based on molecular fingerprints from known chemical structures. The models also demonstrate effectiveness in predictivity when validated on SIRIUS predicted molecular fingerprints from MassBank spectra. Furthermore, a user-friendly web app was created to facilitate interaction with the data and the MLinvitroTox framework.
\end{abstract}
